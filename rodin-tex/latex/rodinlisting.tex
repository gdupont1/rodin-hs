% ------------------------------------------------------------------------
% rodinlisting.tex
% ------------------------------------------------------------------------
%  Copyright (C) 2019  G. Dupont
%  
%  This program is free software: you can redistribute it and/or modify
%  it under the terms of the GNU General Public License as published by
%  the Free Software Foundation, either version 3 of the License, or
%  (at your option) any later version.
%  
%  This program is distributed in the hope that it will be useful,
%  but WITHOUT ANY WARRANTY; without even the implied warranty of
%  MERCHANTABILITY or FITNESS FOR A PARTICULAR PURPOSE.  See the
%  GNU General Public License for more details.
%  
%  You should have received a copy of the GNU General Public License
%  along with this program.  If not, see <https://www.gnu.org/licenses/>.
% ------------------------------------------------------------------------
% This file is a complimentary LaTeX file that should be included in any
% file generated using the rodinapi-tex package API.
% It provides the additional definitions needed for compiling documents.
% ------------------------------------------------------------------------
\usepackage{listings}
\usepackage{amsmath}
\usepackage{amsfonts}
\usepackage{amssymb}
\usepackage{mathtools}

%% Special macros for representing Event-B's "weird arrows"
\newcommand{\leftleftrightarrow}{\leftarrow\mkern-14mu\leftrightarrow}
\newcommand{\leftrightrightarrow}{\leftrightarrow\mkern-14mu\rightarrow}
\newcommand{\leftleftrightrightarrow}{\leftrightarrow\mkern-14mu\leftrightarrow}
\newcommand{\lhdminus}{\lhd\mkern-14mu-}
\newcommand{\rhdminus}{\rhd\mkern-14mu-}
\newcommand{\lhdplus}{\lhd\mkern-9mu-}
\newcommand{\partialrightarrow}{\mkern6mu\mapstochar\mkern-6mu\rightarrow}
\newcommand{\partialrightarrowtail}{\mkern9mu\mapstochar\mkern-9mu\rightarrowtail}
\newcommand{\partialtwoheadrightarrow}{\mkern6mu\mapstochar\mkern-6mu\twoheadrightarrow}
\newcommand{\twoheadrightarrowtail}{\rightarrowtail\mkern-18mu\twoheadrightarrow}

%% Event-B language definition
\lstdefinelanguage{eventb}{
    morekeywords={
        end,
        context,
        extends,
        constants,
        sets,
        axioms,
        machine,
        refines,
        sees,
        variables,
        invariants,
        variants,
        events,
        then,
        any,
        theory,
        theories,
        import,
        projects,
        with,
        where,
        type,
        parameters,
        axiomatic,
        definitions,
        operators,
        datatypes,
        datatype,
        args,
        data,
        types,
        arguments,
        constructors,
        theorems,
        when,
        constructors,
        well-definedness,
        condition,
        direct,
        definition,
        recursive,
        case,
        cases,
        proof,
        rules,
        metavariables,
        rewrite,
        rewrites,
        inference,
        given,
        infer
    },
    sensitive=false,
    morecomment=[l]{--},
    alsoletter=-
}




